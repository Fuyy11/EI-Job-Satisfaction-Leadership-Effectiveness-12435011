\documentclass[
  man,
  longtable,
  nolmodern,
  notxfonts,
  notimes,
  colorlinks=true,linkcolor=blue,citecolor=blue,urlcolor=blue]{apa7}

\usepackage{amsmath}
\usepackage{amssymb}




\RequirePackage{longtable}
\RequirePackage{threeparttablex}

\makeatletter
\renewcommand{\paragraph}{\@startsection{paragraph}{4}{\parindent}%
	{0\baselineskip \@plus 0.2ex \@minus 0.2ex}%
	{-.5em}%
	{\normalfont\normalsize\bfseries\typesectitle}}

\renewcommand{\subparagraph}[1]{\@startsection{subparagraph}{5}{0.5em}%
	{0\baselineskip \@plus 0.2ex \@minus 0.2ex}%
	{-\z@\relax}%
	{\normalfont\normalsize\bfseries\itshape\hspace{\parindent}{#1}\textit{\addperi}}{\relax}}
\makeatother




\usepackage{longtable, booktabs, multirow, multicol, colortbl, hhline, caption, array, float, xpatch}
\setcounter{topnumber}{2}
\setcounter{bottomnumber}{2}
\setcounter{totalnumber}{4}
\renewcommand{\topfraction}{0.85}
\renewcommand{\bottomfraction}{0.85}
\renewcommand{\textfraction}{0.15}
\renewcommand{\floatpagefraction}{0.7}

\usepackage{tcolorbox}
\tcbuselibrary{listings,theorems, breakable, skins}
\usepackage{fontawesome5}

\definecolor{quarto-callout-color}{HTML}{909090}
\definecolor{quarto-callout-note-color}{HTML}{0758E5}
\definecolor{quarto-callout-important-color}{HTML}{CC1914}
\definecolor{quarto-callout-warning-color}{HTML}{EB9113}
\definecolor{quarto-callout-tip-color}{HTML}{00A047}
\definecolor{quarto-callout-caution-color}{HTML}{FC5300}
\definecolor{quarto-callout-color-frame}{HTML}{ACACAC}
\definecolor{quarto-callout-note-color-frame}{HTML}{4582EC}
\definecolor{quarto-callout-important-color-frame}{HTML}{D9534F}
\definecolor{quarto-callout-warning-color-frame}{HTML}{F0AD4E}
\definecolor{quarto-callout-tip-color-frame}{HTML}{02B875}
\definecolor{quarto-callout-caution-color-frame}{HTML}{FD7E14}

%\newlength\Oldarrayrulewidth
%\newlength\Oldtabcolsep


\usepackage{hyperref}




\providecommand{\tightlist}{%
  \setlength{\itemsep}{0pt}\setlength{\parskip}{0pt}}
\usepackage{longtable,booktabs,array}
\usepackage{calc} % for calculating minipage widths
% Correct order of tables after \paragraph or \subparagraph
\usepackage{etoolbox}
\makeatletter
\patchcmd\longtable{\par}{\if@noskipsec\mbox{}\fi\par}{}{}
\makeatother
% Allow footnotes in longtable head/foot
\IfFileExists{footnotehyper.sty}{\usepackage{footnotehyper}}{\usepackage{footnote}}
\makesavenoteenv{longtable}

\usepackage{graphicx}
\makeatletter
\newsavebox\pandoc@box
\newcommand*\pandocbounded[1]{% scales image to fit in text height/width
  \sbox\pandoc@box{#1}%
  \Gscale@div\@tempa{\textheight}{\dimexpr\ht\pandoc@box+\dp\pandoc@box\relax}%
  \Gscale@div\@tempb{\linewidth}{\wd\pandoc@box}%
  \ifdim\@tempb\p@<\@tempa\p@\let\@tempa\@tempb\fi% select the smaller of both
  \ifdim\@tempa\p@<\p@\scalebox{\@tempa}{\usebox\pandoc@box}%
  \else\usebox{\pandoc@box}%
  \fi%
}
% Set default figure placement to htbp
\def\fps@figure{htbp}
\makeatother


% definitions for citeproc citations
\NewDocumentCommand\citeproctext{}{}
\NewDocumentCommand\citeproc{mm}{%
  \begingroup\def\citeproctext{#2}\cite{#1}\endgroup}
\makeatletter
 % allow citations to break across lines
 \let\@cite@ofmt\@firstofone
 % avoid brackets around text for \cite:
 \def\@biblabel#1{}
 \def\@cite#1#2{{#1\if@tempswa , #2\fi}}
\makeatother
\newlength{\cslhangindent}
\setlength{\cslhangindent}{1.5em}
\newlength{\csllabelwidth}
\setlength{\csllabelwidth}{3em}
\newenvironment{CSLReferences}[2] % #1 hanging-indent, #2 entry-spacing
 {\begin{list}{}{%
  \setlength{\itemindent}{0pt}
  \setlength{\leftmargin}{0pt}
  \setlength{\parsep}{0pt}
  % turn on hanging indent if param 1 is 1
  \ifodd #1
   \setlength{\leftmargin}{\cslhangindent}
   \setlength{\itemindent}{-1\cslhangindent}
  \fi
  % set entry spacing
  \setlength{\itemsep}{#2\baselineskip}}}
 {\end{list}}
\usepackage{calc}
\newcommand{\CSLBlock}[1]{\hfill\break\parbox[t]{\linewidth}{\strut\ignorespaces#1\strut}}
\newcommand{\CSLLeftMargin}[1]{\parbox[t]{\csllabelwidth}{\strut#1\strut}}
\newcommand{\CSLRightInline}[1]{\parbox[t]{\linewidth - \csllabelwidth}{\strut#1\strut}}
\newcommand{\CSLIndent}[1]{\hspace{\cslhangindent}#1}





\usepackage{newtx}

\defaultfontfeatures{Scale=MatchLowercase}
\defaultfontfeatures[\rmfamily]{Ligatures=TeX,Scale=1}





\title{Leader's Emotional Intelligence and Job Satisfaction: The
Mediating Effect of Leadership Effectiveness}


\shorttitle{Emotional Intelligence and Job Satisfaction: Leadership
Effectiveness}


\usepackage{etoolbox}






\author{Yangyang Fu}



\affiliation{
{Department of Social Science, University of Chicago}}




\leftheader{Fu}



\abstract{Purpose: This study aimed to determine the effect of leaders'
emotional intelligence on employee job satisfaction and explore the
mediating role of leadership effectiveness.

Methodology: A cross-sectional research design was utilized to study the
research objectives. A questionnaire survey collected data from interns
and employees, totaling 232 (100 males, 131 females, 1 prefer not to
tell). Emotional Intelligence, Job Satisfaction, and Leadership
Effectiveness Scales were utilized to measure the three variables.

Findings: The findings declared that leaders' emotional intelligence
could predict leadership effectiveness and employee job satisfaction,
and leadership effectiveness could also predict job satisfaction.
Moreover, leadership effectiveness partially mediated the mechanism of
leaders' emotional intelligence to employee job satisfaction.

Significances: This research gives some insights into the importance of
leaders' emotional intelligence within an organization or a team. It may
help organizations or teams to improve employee job satisfaction and
enhance efficiency and effectiveness.}

\keywords{Leaders' emotional intelligence, leadership effectiveness, job
satisfaction}

\authornote{\par{\addORCIDlink{Yangyang Fu}{0000-0000-0000-0001}} 
\par{ }
\par{   The authors have no conflicts of interest to disclose.    }
\par{Correspondence concerning this article should be addressed
to Yangyang Fu, Department of Social Science, University of
Chicago, 5801 South Ellis
Avenue, Chicago, IL 60637, USA, Email: fyy@uchicago.edu}
}

\makeatletter
\let\endoldlt\endlongtable
\def\endlongtable{
\hline
\endoldlt
}
\makeatother
\RequirePackage{longtable}
\DeclareDelayedFloatFlavor{longtable}{table}

\urlstyle{same}



\makeatletter
\@ifpackageloaded{caption}{}{\usepackage{caption}}
\AtBeginDocument{%
\ifdefined\contentsname
  \renewcommand*\contentsname{Table of contents}
\else
  \newcommand\contentsname{Table of contents}
\fi
\ifdefined\listfigurename
  \renewcommand*\listfigurename{List of Figures}
\else
  \newcommand\listfigurename{List of Figures}
\fi
\ifdefined\listtablename
  \renewcommand*\listtablename{List of Tables}
\else
  \newcommand\listtablename{List of Tables}
\fi
\ifdefined\figurename
  \renewcommand*\figurename{Figure}
\else
  \newcommand\figurename{Figure}
\fi
\ifdefined\tablename
  \renewcommand*\tablename{Table}
\else
  \newcommand\tablename{Table}
\fi
}
\@ifpackageloaded{float}{}{\usepackage{float}}
\floatstyle{ruled}
\@ifundefined{c@chapter}{\newfloat{codelisting}{h}{lop}}{\newfloat{codelisting}{h}{lop}[chapter]}
\floatname{codelisting}{Listing}
\newcommand*\listoflistings{\listof{codelisting}{List of Listings}}
\makeatother
\makeatletter
\makeatother
\makeatletter
\@ifpackageloaded{caption}{}{\usepackage{caption}}
\@ifpackageloaded{subcaption}{}{\usepackage{subcaption}}
\makeatother

% From https://tex.stackexchange.com/a/645996/211326
%%% apa7 doesn't want to add appendix section titles in the toc
%%% let's make it do it
\makeatletter
\xpatchcmd{\appendix}
  {\par}
  {\addcontentsline{toc}{section}{\@currentlabelname}\par}
  {}{}
\makeatother

%% Disable longtable counter
%% https://tex.stackexchange.com/a/248395/211326

\usepackage{etoolbox}

\makeatletter
\patchcmd{\LT@caption}
  {\bgroup}
  {\bgroup\global\LTpatch@captiontrue}
  {}{}
\patchcmd{\longtable}
  {\par}
  {\par\global\LTpatch@captionfalse}
  {}{}
\apptocmd{\endlongtable}
  {\ifLTpatch@caption\else\addtocounter{table}{-1}\fi}
  {}{}
\newif\ifLTpatch@caption
\makeatother

\begin{document}

\maketitle


\setcounter{secnumdepth}{-\maxdimen} % remove section numbering

\setlength\LTleft{0pt}


\section{Introduction}\label{introduction}

Job satisfaction is highly researched in industrial/organizational
psychology and is related to the development of organizations worldwide.
It is also noted as a predictor for assessing the efficiency and
effectiveness of organizations. Irabor and Okolie
(\citeproc{ref-Irabor2019}{2019}) refer to job satisfaction as
employees' emotional response to a job, which could be viewed as the sum
of contentment and pleasure. Irabor and Okolie
(\citeproc{ref-Irabor2019}{2019}) indicated that job satisfaction can
influence attendance rate, productivity, organizational loyalty, and
life satisfaction. Low employee job satisfaction would result in a
higher possibility of avoiding the workplace and decreased productivity
and quality. The accumulation of negative emotions keeps increasing and
ultimately accounts for turnover.

Helm (\citeproc{ref-Helm2013}{2013}) claimed that job satisfaction
correlated with employees' pride and organizational reputation. High job
satisfaction in the organization could positively influence employees'
sense of group honor and assist in fostering the organization's
reputation, attracting excellent elites. Furthermore, job satisfaction
was strongly associated with mental health problems Faragher et al.
(\citeproc{ref-Faragher2005}{2005}), which means satisfied employees are
less likely to be overstressed and anxious.

Employee job satisfaction could be impacted by satisfaction with pay,
promotion opportunities, and fringe benefits, no matter monetary or
nonmonetary reward, organizational rules, and so on Kardam and Rangnekar
(\citeproc{ref-Kardam2012}{2012}). Organizational culture is one of the
factors that influence employees' job satisfaction Lund
(\citeproc{ref-Lund2003}{2003}). Organizational values or norms shape or
change individuals' behaviors, imperceptibly or compulsively. For
example, Thekedam (\citeproc{ref-Thekedam2010}{2010}) mentioned that
recognition from supervisors or coworkers, no matter blame or praise, is
a crucial factor influencing job satisfaction. Recognition programs can
contribute to maintaining employees' self-esteem and enthusiasm at high
levels Danish and Usman (\citeproc{ref-Danish2010}{2010}). It could
fulfill employees' psychological needs within the workplace, thus
increasing their motivation. Statista Research Department
(\citeproc{ref-BambooHR2016found}{\textbf{BambooHR2016found?}}) that
over 80\% of employees feel satisfied working in companies with a
recognition program, whether formal or informal, whose satisfaction rate
is 30\% higher than those working for companies with no recognition
program. A significant component of the organizational culture is
leadership behaviors, including trustiness, role modeling, and
communication, which also showed a positive relationship with employees'
outcomes and satisfaction Chiok Foong Loke
(\citeproc{ref-ChiokFoongLoke2001}{2001}).

Leadership behaviors are directly associated with Emotional Intelligence
(EI), and the degree of EI in a leader can shape the leadership
behavior. As the complexity of the relationships between individuals and
organizations rises, EI also begins to be recognized as a significant
personality trait for workplace success BambooHR
(\citeproc{ref-BambooHR2016}{2016}). Moreover, EI skills are utilized as
criteria for many organizations in the Western world to achieve career
selection and placement purposes, thus becoming a crucial element of
management philosophy Zeidner et al. (\citeproc{ref-Zeidner2004}{2004}).
Yadav and Lata (\citeproc{ref-Yadav2019}{2019}) defined EI as an ability
to notice and understand other's emotions and apply this information to
influence people's behavior and thinking. It implies that people with
strong EI can handle interpersonal conflicts, help find potential
problems, and provide emotional support Atoum and Al-Shoboul
(\citeproc{ref-Atoum2018}{2018}); Tanveer et al.
(\citeproc{ref-Tanveer2019}{2019}). Employees' job satisfaction is
considered one of the outcomes of leaders' EI influence.

Moreover, leadership effectiveness is regarded as leaders' effects on
the group's performance on the tasks Levine and Moreland
(\citeproc{ref-Levine2006}{2006}). Leaders with high EI have solid
social skills to navigate complex interpersonal relationships and assign
functions to proper employees, meaning they may influence their
leadership effectiveness. In addition, leadership effectiveness, whose
one indicator is the performance management system Rosete and Ciarrochi
(\citeproc{ref-Rosete2005}{2005-07-01Z}), may related to employees'
attendance and their attitude to organizational rules and procedures.
For example, a leader able to handle their own and employees' emotions
would be able to solve interpersonal conflicts effectively and help
build up work achievability, thus demonstrating high effectiveness.
Therefore, these links suggest that the link between a leader's EI and
an employee's Job Satisfaction can be explained by leadership
effectiveness.

We know that recognition Thekedam (\citeproc{ref-Thekedam2010}{2010}),
organizational culture Lund (\citeproc{ref-Lund2003}{2003}), and
leadership behavior Chiok Foong Loke
(\citeproc{ref-ChiokFoongLoke2001}{2001}) can exert impacts on
employees' job satisfaction in terms of different angles. Some studies
investigate how employees' EI influences job satisfaction in Western and
European countries. There is still limited information, and data about
leaders' EI in the context of Chinese background is scarce. Liu et al.
(\citeproc{ref-Liu2018}{2018}) argued that Chinese national culture is
immersed in Confucian philosophy, emphasizing people and relationship
focus, influencing organizations' operations and conduction. Compared to
Western countries, relationships are particularly crucial within Chinese
organizations, which offer a significant position for EI. Furthermore,
relationship governance and definitions are not identical in business
between China and Western countries. According to
(\citeproc{ref-Chu2020}{\textbf{Chu2020?}}), the ``relationship'' in
Chinese is also named Guanxi, which is usually nurtured and facilitated
through conversations, meetings, and meals started by Chinese people or
organizations, leveraging connections based on mutual obligations and
exchanging favors. In contrast, Western organizations focus on trust and
commitment, relying on self-enforce and avoiding opportunistic behavior
(\citeproc{ref-Chu2020}{\textbf{Chu2020?}}). In Guanxi, affection is
essential in organizational behavior and employee commitment,
influencing job satisfaction. Affection is a key feature differentiating
Chinese culture from other cultures
(\citeproc{ref-Wu2020}{\textbf{Wu2020?}}).
(\citeproc{ref-Wu2020}{\textbf{Wu2020?}}) also discovered that employee
affective commitment could positively affect decreasing elements,
including turnover and absenteeism rates that reflect job satisfaction.
Hence, exploring the influence of leaders' EI within Chinese cultural
contexts is essential to a meaningful endeavor to understand to what
degree the EI of leaders in the Chinese context leads to leadership
effectiveness as an underlying mechanism for employee job satisfaction.
Understanding the effect of leaders' EI could help determine how to
improve the organization's promotion system and increase the
organization's efficiency and effectiveness in the Chinese context.
Moreover, leadership development programs could be designed to increase
leaders' self-awareness and fix shortcomings to increase leadership
effectiveness in line with the Chinese culture, increasing employees'
job satisfaction and avoiding adverse impacts such as high turnover.
Thus, the current study aims to investigate how leaders' EI predicts
employees' job satisfaction and understand the mediating effects of
leadership effectiveness between the two variables in the Chinese
context.

The research question of this study is :

Are there any mediating effects of leadership effectiveness between
leaders' emotional intelligence and employee job satisfaction?

This research seeks to answer this question based on a specific cultural
background. It aims to examine the direct or indirect effects of
leaders' EI on job satisfaction through the mediating role of leadership
effectiveness.

Therefore, the study hypotheses that:

H1: Leaders' emotional intelligence will predict employee job
satisfaction.

H2: Leaders' emotional intelligence will predict Leadership
effectiveness.

H3: Leadership effectiveness will predict employee job satisfaction.

H4: Leadership effectiveness will mediate leaders' emotional
intelligence and job satisfaction.

\section{Literature Review}\label{literature-review}

\subsection{Job Satisfaction}\label{job-satisfaction}

Job Satisfaction is widely researched by many researchers, whether in
industrial organization psychology or management, due to its importance
to employees' mental health and organizational development. Cumbey and
Alexander (\citeproc{ref-Cumbey1998}{1998}) described job satisfaction
as a feeling depending on the interplay between employees and personal
traits, values, and expectations. Locke (\citeproc{ref-Locke1969}{1969})
specifically defined job satisfaction as a positive emotional state
arising when someone agrees to achieve their values and goals in the
workplace; job dissatisfaction was negative feelings resulting from
being unable to attain their values and goals. Aziri
(\citeproc{ref-Aziri2011}{2011}) also defined job satisfaction as a
combination of feelings and beliefs about their current work, which
ranges between extremes (extreme satisfaction and extreme
dissatisfaction). An encompassing definition of job satisfaction can be
seen as feelings about work experience, including fulfillment and
engagement, a working environment, and interpersonal relationships.

Numerous studies relate job satisfaction with employee performance; for
example, Inuwa (\citeproc{ref-Inuwa2016}{2016}) found a significant
positive relationship between employee satisfaction and employee
performance. The study found that an increase in job satisfaction of
employees can directly lead to higher performance.
(\citeproc{ref-Aziri2011also}{\textbf{Aziri2011also?}}) gave similar
results, which implies that employees who are satisfied due to some
recognition and rewards are more likely to lead to more significant
performance effort and achieve a higher level of job performance.
Moreover, Khan et al. (\citeproc{ref-Khan2012}{2012}) also found that
high-level job satisfaction can predict better employee performance in
Pakistan. However, little research has been conducted on employees' job
satisfaction in the Chinese context. The last four decades of research
before 2009 on job satisfaction and work outcomes were based on Western
culture, and Han and @ Kakabadse2009 also highlighted that job
satisfaction was closely related to culture, such as Hofstede's six
cultural dimensions, which means specific cultural contexts would have
some impacts, such as deeply rooted Confucianism perspective Zhang et
al. (\citeproc{ref-Zhang2019}{2019年9月26日}) In addition, according to
Hofstede's theory about six cultural dimensions, China has a high score
in power distance, which may lead to downward pressure on employees and
the ignoring of leaders themselves. Lu et al.
(\citeproc{ref-Lu2011}{2011}) said that heavy workload and especially
interpersonal conflict play an important role in work stress in the
Greater China Zone. Zhang et al.
(\citeproc{ref-Zhang2019}{2019年9月26日}) explained that a
Confucian-based work ethic characterized by hard work, endurance,
collectivism, and ``guanxi'' (personal network) expected Chinese
employees to take full responsibility, promote social harmony, and
advance collective interests. The working environment produced under the
cultivation of Chinese culture leads to different expectations from
organizations and leaders, as well as different working standards in
employees' hearts. A past study conducted by Luu and Hattrup
(\citeproc{ref-Luu2010}{2010}) found that high job satisfaction is
related to low turnover intention in America. However, some researchers,
such as Wong et al. (\citeproc{ref-Wong2001}{2001}), indicated that job
satisfaction can't effectively predict turnover intention within China.
Therefore, looking into the Chinese context could help to figure out the
feelings of employees and develop customized programs to improve
conditions. There are five primary dimensions of job satisfaction,
including salaries, promotion opportunities, supervision, nature of
work, and colleagues, according to the research by Aslaniyan and
Moghaddam (\citeproc{ref-Aslaniyan2013}{2013}). They clarify that pay
satisfaction is related to feelings of deserving and worthy; promotion
satisfaction is linked to organizations' policy, fairness, and
employees' ability; supervision and colleagues mean their feelings for
bosses and co-workers; the nature of work represents characteristics of
work itself, including challenge, interest, and comfort Aslaniyan and
Moghaddam (\citeproc{ref-Aslaniyan2013}{2013}). Waqas et al.
(\citeproc{ref-Waqas2014}{2014}) also mentioned that reward and
recognition and the workplace environment can also influence it,
respectively contributing to 49.8\% and 25.7\% of job satisfaction
levels. A reward as a tangible gift and recognition are both used to
acknowledge the public or organization's employees' contribution Waqas
et al. (\citeproc{ref-Waqas2014}{2014}), which can motivate employees
and increase their job satisfaction, whether mentally or physically.
Working environment, including lighting, temperature, workspace, and
ambient conditions, can influence employees' perception of comfort Waqas
et al. (\citeproc{ref-Waqas2014}{2014}). The present study will examine
job satisfaction in the Chinese context to figure out how it would be
influenced by leaders' EI and find ways to improve the current situation
of low job satisfaction in China.

\subsection{Leader's Emotional
Intelligence}\label{leaders-emotional-intelligence}

EI is considered a significant tool for interacting with people, whether
in a small conversation or on other essential occasions. In 2019, more
than 50\% of organizations respondents worldwide indicated that they
would evaluate senior and mid-management EI skills and choose to hire
them or not based on it Capgemini (\citeproc{ref-Capgemini2019}{2019}).
The high percentage of EI measures used reveals the growing importance
of EI in organizational management. The high percentage of EI measures
used reveals the growing importance of EI in organizational management.
According to Yadav and Lata (\citeproc{ref-Yadav2019}{2019}), EI has
already been established in some non-cognitive metrics, including
problem-solving skills. They also summarize
(\citeproc{ref-Goleman1999definition}{\textbf{Goleman1999definition?}})
of EI into four kinds of abilities: zeal, self-control, self-motivation,
and persistence, characterized by understanding, managing, and
recognizing emotions, and handling motivations and relationships Yadav
and Lata (\citeproc{ref-Yadav2019}{2019}). Mayer et al.
(\citeproc{ref-Mayer1990}{1990}) divided EI into four branches as well,
including appraising and perceiving, assimilating, understanding, and
regulating emotions. Here is a tricky point Ramesar et al.
(\citeproc{ref-Ramesar2009}{2009}) suggested about emotion concepts:
emotions are shorter and more intense than moods. Martinez
(\citeproc{ref-Martinez1997}{1997}) defined EI more concisely into
non‐cognitive skills, capabilities, and competencies, which determine
how people manage and respond to environmental challenges and pressure.
Most definitions of EI talk about the ability to manage and sense
emotions instead of moods, converting them into effective actions and
interactions.

Some researchers found that EI correlated with people's stress
management, which could be considered as a component, input, and even
outflow Ramesar et al. (\citeproc{ref-Ramesar2009}{2009}), and it can
positively influence and help with finding effective strategies for
coping with stress Ramesar et al. (\citeproc{ref-Ramesar2009}{2009});
Wang et al. (\citeproc{ref-Wang2016}{2016}). According to Wang et al.
(\citeproc{ref-Wang2016}{2016}), college students with high EI have a
great positive effect and can regulate their emotions better, leading to
increased self-confidence and improved ability to handle pressure in
situations. A study that used qualitative unstructured depth interviews
done by Pau et al. (\citeproc{ref-Pau2004}{2004}) indicated that
students who have a higher level of EI can cope with stress better by
utilizing skills of self-reflection and assessment, social and
interpersonal interactions, as well as effective organization and
time-management abilities. The research assessed students' EI and coping
strategies through their answers in the conversation and discovered that
students with high EI can tell whether their behaviors and actions are
appropriate. Additionally,
(\citeproc{ref-Ivcevic2021}{\textbf{Ivcevic2021?}}) declared that
leaders' emotional intelligence behavior is one of the job sources,
motivating and resonating with employees emotionally and ultimately
effectively predicting employee's creativity behaviors. Darvishmotevali
et al. (\citeproc{ref-Darvishmotevali2018}{2018}) also found that EI had
a significant impact on employees' creativity performance. Regarding the
relationship between EI and creativity,
(\citeproc{ref-Geher2017explained}{\textbf{Geher2017explained?}}) that
the ability to feel others' feelings accurately finds what is amusing
for people, which also means people who have high EI are capable of
jumping out of the box.

In common research, EI is usually measured by self-report scales using
quantitative methods.
(\citeproc{ref-AugustoLanda2008did}{\textbf{AugustoLanda2008did?}}) a
questionnaire survey to investigate the relationships between EI, work
stress, and health. Self-report scales were used to evaluate
participants' EI. They found that EI serves as a protective element
against stress and promotes overall health and well-being Augusto Landa
et al. (\citeproc{ref-AugustoLanda2008}{2008}). However, the accuracy of
the EI self-report is questioned. Rosete and Ciarrochi
(\citeproc{ref-Rosete2005}{2005-07-01Z}) insisted that self-report
scales are easily influenced by respondents' personalities, leading to
some confusion. The domain research about EI uses self-reporting to
assess, but little research has been conducted to study leaders' EI
through others' observations. Moreover,@ Gökçen2014 said East Asian
individuals typically tend to show self-criticism and emphasize negative
self-relevant information compared to North Americans. Self-perception,
as a crucial aspect of EI, influenced by cultural differences and
tendencies, can be essential to outcomes in China compared to different
countries. That's the reason why the present study aims to discover how
leaders' EI is evaluated by subordinates' work and how East Asian
special perspectives would influence the process compared to Western
countries. Factors including personal factors that influence EI are
still controversial Rauf et al. (\citeproc{ref-Rauf2013}{2013}); for
example, Van Rooy et al. (\citeproc{ref-VanRooy2005}{2005}) found there
was a significant difference between males and females, but others held
opposite ideas Katyal and Awasthi (\citeproc{ref-Katyal2017}{2017}).
Rauf et al. (\citeproc{ref-Rauf2013}{2013}) discovered that their study
showed only family monthly income, not parents' employment and parental
educational level, has a positive relationship with EI. However, it is
inconsistent with the research done by Davis-Kean
(\citeproc{ref-Davis-Kean2005}{2005}), which supported that parents'
educational background can positively influence children's EI. This
study investigates leaders' EI through others' evaluation and
observation in the context of the Chinese working environment and
identifies its effect on leadership effectiveness and subordinates.
Moreover, it could help organizations improve leaders' selection systems
and facilitate the forming of more suitable and efficient teams.

\subsection{Leadership Effectiveness}\label{leadership-effectiveness}

Leadership Effectiveness is one of the important elements in evaluating
leadership and outcomes, and it's also a determined factor for teams or
organizations in achieving success. Leadership effectiveness can be
understood as successful outcomes in achieving common goals or missions
from leaders' influence. Sadeghi et al.
(\citeproc{ref-Sadeghi2012}{2012}) said it highly depends on the
consequences and impacts of leaders' activities on their followers,
teams, or organizations. Many studies have found that leadership
effectiveness was correlated with leadership styles. Sadeghi et al.
(\citeproc{ref-Sadeghi2012}{2012}) claimed that transformational
leadership behavior can motivate followers and have positive impacts on
the organization's effectiveness and productivity.
(\citeproc{ref-SonmezCakir2020}{\textbf{SonmezCakir2020?}}) argued that
leadership effectiveness is a wavier behavior that is exchanged for
organizations' benefits, which can inspire employees' behaviors and
shape their goals through enhancing employees' commitments and
motivations, thus leading to a positive impact on organizations.
Furthermore, leadership effectiveness can influence engaging employees'
``beliefs, attitudes, norms, values, and behaviors'' in the long-term
(\citeproc{ref-SonmezCakirAdiguzel2020}{\textbf{SonmezCakirAdiguzel2020?}}),
p.~2. According to Rawat et al. (\citeproc{ref-Rawat2023}{2023}),
efficient leadership from successful leadership can foster and
facilitate a sense of confidence and independence within teams, ensuring
they can cope with dynamic environments. It could also help form
adaptive resilience, including changing plans to adapt, thrive, and
recover during the crisis Rawat et al. (\citeproc{ref-Rawat2023}{2023}).
Personality traits are also proven to have an influence on leadership
effectiveness, such as stress tolerance, socialized power motivation,
achievement orientation, and need for affiliation Hartman
(\citeproc{ref-Hartman1999}{1999}). Some metrics, including followers'
attitude, group process quality, group cohesiveness, etc., are usually
used to evaluate leadership effectiveness Hartman
(\citeproc{ref-Hartman1999}{1999}). In addition, leadership dimensions,
in other words, leadership styles such as transformational,
transactional, and laissez-faire styles, exert different impacts on
leadership effectiveness Yahaya and Ebrahim
(\citeproc{ref-Yahaya2016}{2016}). Specifically, Laissez-faire leaders
tend to ignore issues, avoid decision-making, and refuse to follow
through and intervene, ultimately leading to a lack of leadership
effectiveness Yahaya and Ebrahim (\citeproc{ref-Yahaya2016}{2016}). In
contrast, Yahaya and Ebrahim (\citeproc{ref-Yahaya2016}{2016}) realized
that the other two leadership styles can positively influence leadership
effectiveness to different degrees. Poturak et al.
(\citeproc{ref-Poturak2020}{2020}) indicated that the transformational
leadership dimension could positively increase leadership effectiveness
through idealized influence, inspirational motivation, intellectual
stimulation, and individual consideration. However, transactional
leadership utilizes rewards or sanctions as incentives to motivate
individuals to pursue self-interest and simultaneously achieve
organizations' or teams' goals Jacobsen et al.
(\citeproc{ref-Jacobsen2022}{2022}). Factors that influence leadership
effectiveness also include communication competency Schuetz
(\citeproc{ref-Schuetz2017}{2017}) and the language of leadership
Lowenhaupt (\citeproc{ref-Lowenhaupt2014}{2014}). Schuetz
(\citeproc{ref-Schuetz2017}{2017}) claimed that communication competency
plays an important role when leaders try to set visions and directions
for a team or organization. Nguyen et al.
(\citeproc{ref-Nguyen2022}{2022}) indicated that communication skills
are linked to emotional intelligence, and they assist each other in
formulating a comfortable environment and maintaining group
relationships, achieving harmony and unity of purpose. To sum up,
leadership effectiveness is a critical factor in evaluating leaders' or
teams' outcomes, and it is important for researchers to investigate
whether it has mediation effects on EI and job satisfaction.

Vilkinas et al. (\citeproc{ref-Vilkinas2009}{2009}) argued that China's
cultural and institutional environment differs from the development
environment of Western dominant leadership theory. They also argued that
these differences significantly differ in leadership behaviors related
to leadership effectiveness Vilkinas et al.
(\citeproc{ref-Vilkinas2009}{2009}). Hassan et al.
(\citeproc{ref-Hassan2018}{2018}) indicated that context and leaders'
skills are still critical issues when discussing leadership
effectiveness. Furthermore, many past studies invested leadership
effectiveness as an independent or dependent variable or directly seen
as an adjunct of leadership. Little research makes detailed descriptions
of leadership effectiveness or perceives it as a mediator between two
other factors. That's the reason why this present study aims to examine
the mediation effects of leadership effectiveness in Chinese culture.
Interestingly, Hassan et al. (\citeproc{ref-Hassan2018}{2018}) found
that the literature on the relationship between leadership effectiveness
and organizational performance is controversial and full of differences
and divergent findings. Inconsistent findings bring a big challenge to
exploring the impacts of leadership effectiveness.

\section{Method}\label{method}

The current study is a cross-sectional study aiming to investigate the
relationship between leaders' emotional intelligence and employee job
satisfaction, determining the mediating effects of leadership
effectiveness on them. The reasons for conducting a cross-sectional
survey are its advantages, including flexibility, time-saving, and lower
costs Connelly (\citeproc{ref-Connelly2016}{Sep/Oct 2016}), which could
help to collect data efficiently within months. In addition, the study
aims to discover variables in a single period of time, and a
cross-sectional study is suitable for this purpose. \#\# Participants A
sample of 232 respondents participated in the research, including
interns and official employees from several industries, including
education, Manufacturing, Construction, the Internet, Electric gas, and
many other sectors working in China. In addition, the present study
chose a non-probability form of sampling: a convenience sampling
technique characterized by economy, convenience, and simpleness Stratton
(\citeproc{ref-Stratton2021}{2021}). Moreover, among the participants,
57.6\% were female, and only one person preferred not to tell; 34.9\%
were interns, 21.4\% were from the education industry, and all
participants were over 18.

\subsection{Measures}\label{measures}

\subsubsection{Leaders' Emotional Intelligence
Scale}\label{leaders-emotional-intelligence-scale}

Hu et al. (\citeproc{ref-Hu2023}{2023}) claimed that leaders' EI could
be measured from emotional appraisal of self and others, use of emotion,
and emotional regulation. This scale contained 16 items and four items
for each aspect. The Cronbach's alpha was 0.924, greater than 0.3, which
indicated that the scale could give consistent results Hu et al.
(\citeproc{ref-Hu2023}{2023}). A 5-point Likert scale ranging from
strongly disagree, disagree, neutral, agree, to strongly agree was used
as a tool. Here are some examples: ``My leader has a good sense of why
he has certain feelings most of the time.'' ``My leader always knows his
friends' emotions from their behavior.'' ``My leader always sets goals
for himself and then tries his best to achieve them.'' ``My leader is
able to control his temper and handle difficulties rationally.''

\subsubsection{Leadership Effectiveness
Scale}\label{leadership-effectiveness-scale}

The research measured leadership effectiveness through the scale from
Kerr et al. (\citeproc{ref-Kerr2006}{2006}), which a third party
generated as a 10-point Likert scale. The present research modified it
into a 5-point Likert scale for consistency with other scales. There was
a total of 9 items. Here are some examples: ``I feel at ease with my
supervisor when asking questions'' and ``My supervisor asks me how I am
doing on a regular basis.''

\subsubsection{Job Satisfaction Scale}\label{job-satisfaction-scale}

Employees' job satisfaction would be measured by the 5-point Likert job
satisfaction scale developed by Brayfield and Rothe
(\citeproc{ref-Brayfield1951}{1951}). The reliability coefficient was
0.77, ``which was corrected by the Spearman-Brown formula to 0.87''
Brayfield and Rothe (\citeproc{ref-Brayfield1951}{1951}),p.310. To keep
a consistent result, answers to some questions would be reversed in the
data analysis process. Here are some examples, ``My job is like a hobby
to me.'' ``I am often bored with my job.'' ``I am satisfied with my job
for the time being.''

\subsection{Procedure}\label{procedure}

The data collection was conducted during the 2024 spring semester,
lasting for three months. The survey was designed and published by
utilizing Sojump, a questionnaire platform. It was spread mainly through
social media, including WeChat, Weibo, and Xiaohongshu. The entire
questionnaire took respondents about 6 minutes to finish. The survey
mainly contained three sections. The first section is a consent form
where students are informed that their responses and information are
confidential and will only be used for research. It clearly states,
``Participation in this study is purely voluntary.'' The second section
is the part where participants are asked to rate the questions about
their leaders' emotional leadership, leadership effectiveness, and job
satisfaction. The third section collects demographic information.

\section{Data analysis and results}\label{data-analysis-and-results}

\subsection{Response Rate}\label{response-rate}

The present study distributed questionnaires online, and 232 people
responded. However, three respondents were excluded from the analysis
for giving redundant responses. Among the data, three respondents gave
the same results on all Likert Scale. 229 questionnaires remain for
further data analysis. Thus, the response rate was 98.7\%.

\subsection{Missing Data}\label{missing-data}

Emmanuel et al. (\citeproc{ref-Emmanuel2021}{2021}) claimed that missing
data problems could lead to performance degradation, biased data
analysis, and outcomes. After checking the data, no missing data was
found.

\subsection{Outliers}\label{outliers}

A univariate outlier analysis was used in the present study. According
to (\citeproc{ref-Mowbray2019}{\textbf{Mowbray2019?}}), univariate
outliers are extreme values different from domain cases. Range function
was used to identify errors in data entry. Three responses were found
invalid because the same answers were used for all rating scales, which
were deleted before conducting data analysis.

\subsection{Normality}\label{normality}

In the current study, the Kurtosis and Skewness were conducted,
revealing that for the variable of leaders' emotional intelligence, the
values of Kurtosis and Skewness were -1.081 and .170; for the variable
of leadership effectiveness, the values of Kurtosis and Skewness were
-.924 and .106; for the variable of Employee Job Satisfaction, the
values of Kurtosis and Skewness were -.428 and .659.
(\citeproc{ref-George2010}{\textbf{George2010?}}) indicated that the
values for skewness and kurtosis of variables are suitable when ranging
from -2 to +2. Therefore, the values prove that leaders' emotional
intelligence, leadership effectiveness, and job satisfaction are in a
normal univariate distribution. The values of Kurtosis and Skewness are
shown in Table 1.

\subsection{Descriptive statistics}\label{descriptive-statistics}

According to Marshall and Jonker (\citeproc{ref-Marshall2010}{2010}),
descriptive statistics describe the characteristics of the raw data,
including central tendency, dispersion, mean, median, standard
deviation, and so on. It is an easy and helpful method for researchers
to summarize the sample while assisting in further data analysis. The
descriptive statistics for the three variables in the present study are
shown in Table 1. The mean for leaders' emotional intelligence people
rate was 3.168 (SD=0.85); the mean for leadership effectiveness was
3.165 (SD=0.85); and the mean for job satisfaction was 3.029 (SD=0.79).
Additionally, Table 1 declared that there were positive relationships
between leaders' EI, leadership effectiveness, and employee job
satisfaction.

\subsection{Reliability checks of
scale}\label{reliability-checks-of-scale}

Cortina (\citeproc{ref-Cortina1993}{1993}) indicated that coefficient
alpha is one of the most important and widely used statistics in
different research fields to test scales' reliability and internal
consistency. According to Ursachi et al.
(\citeproc{ref-Ursachi2015}{2015}), it's commonly accepted by the
general when Cronbach alpha ranges from 0.6 to 0.7 and when it comes to
greater than 0.8, suggesting a good reliability and internal
consistency. Table 1 showed that the Cronbach alpha of the leaders'
emotional intelligence scale is 0.916; for leadership effectiveness, it
is 0.844; and for job satisfaction, it is 0.903. The coefficient alpha
for the three scales is acceptable.

\subsection{Final analysis- Hypothesis
Testing}\label{final-analysis--hypothesis-testing}

H1: Leaders' emotional intelligence will predict employee job
satisfaction.

Results (shown in Table 2) discovered that a leader's emotional
intelligence was a significant predictor of job satisfaction (β=0.826;
SE = 0.035; p \textless.001). This indicates that higher leaders'
emotional intelligence can be considered a predictor of higher employee
job satisfaction. In other words, employees working on a team with
leaders with high emotional intelligence tend to report high job
satisfaction.

H2: Leaders' emotional intelligence will predict Leadership
effectiveness.

Results (shown in Table 2) found that leaders' emotional intelligence
was a significant predictor of leadership effectiveness (β = 0.875; SE =
0.032; p \textless{} .001). It means that groups or teams with leaders
who have high emotional intelligence can predict the emergence of
leadership effectiveness. In other words, leaders with higher emotional
intelligence and higher leadership effectiveness may emerge.

H3:Leadership effectiveness will predict employee job satisfaction.

Results (shown in Table 2) indicated that leadership effectiveness
emerged as a significant predictor of employee job satisfaction
(β=0.810, SE = 0.036; p \textless{} .001). It implies that leadership
effectiveness can positively influence employee job satisfaction. In
other words, employees working in an environment that embraces good
leadership effectiveness tend to be satisfied with their jobs.

H4: Leadership effectiveness will mediate leaders' emotional
intelligence and job satisfaction.

To determine whether there is a mediation effect of leadership
effectiveness between leaders' emotional intelligence and employee job
satisfaction, the method proposed by
(\citeproc{ref-Baron1986was}{\textbf{Baron1986was?}}) taken. To be
concise, the simple regression analysis was run separately with IV
(leaders' emotional intelligence) predicting mediator (leadership
effectiveness) and DV (job satisfaction), and mediator predicting DV.
Then, a multiple regression analysis was conducted with all variables,
including IV, mediator, and DV. The connection between the predictor and
dependent variable weakens when the intervening variable is controlled.

The analysis indicated that when leadership effectiveness was entered
into the equation, the decrease in the direct path between emotional
intelligence and job satisfaction was statistically significant (Z=
16.503, p \textless.001). It implies that leadership effectiveness
partially mediates the relationships between leaders' emotional
intelligence and leadership effectiveness. In other words, leadership
effectiveness can strengthen the mechanism of leaders' emotional
intelligence effects on job satisfaction.

\section{Discussion}\label{discussion}

This study aimed to discuss the relationship between leaders' emotional
intelligence and employee job satisfaction and explore the mediating
role of leadership effectiveness between them. The current research
proposed four hypotheses, and the results are discussed below:

H1: Leaders' emotional intelligence will predict employee job
satisfaction:

The results indicated that leaders' EI could predict employee job
satisfaction (β=0.826, p \textless.001); thus, hypothesis 1 was
accepted. The strong relationship between leaders' EI and job
satisfaction suggests that leaders with higher EI are more likely to
foster a positive environment for employees to achieve higher job
satisfaction. In other words, leaders who are better at understanding
and managing emotions can help employees feel fulfilled and happy with
their work. This finding emphasized the importance of leaders' ability
to handle the feelings and needs of their own and others when promoting
employee job satisfaction. The current finding is aligned with the prior
research, showing the considerable prediction of leaders' EI on employee
job satisfaction Miao et al. (\citeproc{ref-Miao2016}{2016}); San Lam
and O'Higgins (\citeproc{ref-SanLam2012}{2012}); Sy et al.
(\citeproc{ref-Sy2006}{2006}). According to Sy et al.
(\citeproc{ref-Sy2006}{2006}), managers with high EI are related to
positive attitudes to the job and altruistic actions, including kindness
and support. Vlachos et al. (\citeproc{ref-Vlachos2013}{2013}) indicated
that managers' altruistic behaviors positively correlate with employee
corporate social responsibility-induced intrinsic and extrinsic
attribution, which could offer employees meanings of the work and
challenge goals, fulfilling their sense of achievement and strengthening
job satisfaction ultimately. We can understand that about half of the
employees and interns in the current study work with leaders who have
high-level EI. They brought positive attitudes to the job, and their
altruistic behaviors assisted employees in achieving fulfillment and
satisfaction.

H2: Leaders' emotional intelligence will predict Leadership
effectiveness.

The results revealed that leaders' EI could predict leadership
effectiveness (β = 0.875, p \textless{} .001); thus, the hypothesis 2
was accepted. The predictive relationship between leaders' EI and
leadership effectiveness indicated that leaders' EI level could forecast
their effectiveness in leadership roles in teams or organizations.
Specifically, leaders who are good at navigating emotions and handling
conflicts and changes are likelier to have efficient and effective
project outcomes. The present findings are aligned with previous
research asserting the positive predictive relationship between leaders'
EI and leadership effectiveness Edelman and van Knippenberg
(\citeproc{ref-Edelman2018}{2018}); Judge et al.
(\citeproc{ref-Judge2004}{2004}); Lone and Lone
(\citeproc{ref-Lone2018}{2018}). According to @ Görgens-Ekermans2021, EI
is correlated with leadership style, primarily transformational
leadership. They also indicated that the path influences leadership
effectiveness by EI involving conflict management skills and coaching
and mentoring behaviors. In addition, Managers could apply EI principles
in the workplace to overcome barriers, address and resolve conflicts,
and finally accomplish team goals and outcomes. As mentioned before
about EI, half of the respondents in the current study experience good
leadership effectiveness. Their leaders might show a good ability to
sense their emotions and help manage conflicts in the workplace.

H3: Leadership effectiveness will predict employee job satisfaction.

The results declared that leadership effectiveness could predict
employee job satisfaction (β=0.810, p \textless{} .001); thus,
hypothesis 3 was accepted. This means that there is a predictive effect
of leadership effectiveness on employee satisfaction with the job,
including happiness and fulfillment. In other words, employees working
with leaders who successfully influence followers to achieve common
goals and missions and gain effective outcomes are likelier to
demonstrate high job satisfaction. This finding parallels the previous
results that claimed that leadership effectiveness can positively
influence and forecast employee job satisfaction Jahro et al.
(\citeproc{ref-Jahro2022}{2022}); Yukl et al.
(\citeproc{ref-Yukl2019}{2019}). Leaders who can lead to high leadership
effectiveness indicate that they have better communication language
Lowenhaupt (\citeproc{ref-Lowenhaupt2014}{2014}); Schuetz
(\citeproc{ref-Schuetz2017}{2017}) and ensure team harmony such as
confidence Rawat et al. (\citeproc{ref-Rawat2023}{2023}), making
employees feel comfortable with the working environment. In the current
study, half of the respondents who are employees and interns found their
leaders have high-quality leadership effectiveness, helping reduce
harmful efforts from factors affecting satisfaction.

H4: Leadership effectiveness will mediate leaders' emotional
intelligence and job satisfaction.

The results asserted that leadership effectiveness could partially
mediate the relationship between leaders' EI and employee job
satisfaction (Z= 16.503, p \textless.001). It suggested that leadership
effectiveness was a tool that facilitated the effects of leaders' EI on
employee job satisfaction. Namely, successful leadership effectiveness,
such as offering a comfortable communication environment, can assist
leaders' EI behavior by giving emotional support, which better satisfies
employees' feelings. Leadership effectiveness involves emotions Kerr et
al. (\citeproc{ref-Kerr2006}{2006}), like EI, which implies that
leadership effectiveness could reflect the EI effect, and EI can affect
leadership effectiveness. However, some outside elements could also
contribute to leadership effectiveness, such as organizational policies
Chan (\citeproc{ref-Chan2002}{2002}). With the synergistic action of
many factors, leadership effectiveness is not only a reflection of EI
but a combination of facilitators. Thus, leadership effectiveness showed
a partial mediation effect and decreased the direct impact of leaders'
EI when it got involved.

\section{Significance}\label{significance}

This study gives some insights for organizations and managers to help
improve employee job satisfaction, achieving higher productivity and
efficiency. The present research recommends that organizations consider
leaders' EI a criterion when evaluating candidates and making promotion
decisions. Additionally, training programs were advocated to be designed
for managers and leaders for the purpose of developing their EI and
related skills. Moreover, this study emphasized the importance of
self-reflection and feedback, which could assist in improving
self-awareness and fixing shortcomings. Besides, organizations and
leaders should ensure the alignment of values and behaviors. Because
values, beliefs, and behavior are highly related to leadership
effectiveness in the long term
(\citeproc{ref-SonmezCakir2020}{\textbf{SonmezCakir2020?}}), their
consistency could help build trust, commitment, and team spirit,
contributing to satisfied employees. In short, regarding recommendations
in the discussion, trying to figure out how to improve leaders' EI and
leadership effectiveness could facilitate the research model that this
study discussed.

\section{Limitations and Future
Directions}\label{limitations-and-future-directions}

Leaders' EI significantly predicted leadership effectiveness and
employee job satisfaction. Additionally, leadership effectiveness was
also performed as a mediator that influences the relationship between
leaders' EI and employee job satisfaction. The findings emphasized a new
mechanism during the pathway of leaders' EI to job satisfaction, which
gives some insights for organizations or teams on how to help improve
employee job satisfaction. As mentioned, EI training programs,
reflection and feedback, and alignment with values are good ways to
facilitate relationships between these three variables.

There are some limitations of this study. First, the current study used
a cross-sectional design. Spector (\citeproc{ref-Spector2019}{2019})
clarified that cross-sectional research has problems with common method
variance and cause-effect relationships. Some occasion factors might
influence respondents, lead to bias, and influence common method
variance Spector (\citeproc{ref-Spector2019}{2019}). For example, a
recent complaint about leaders may influence respondents' judgment,
which serves as the criterion for rating the scales. Additionally,
cross-sectional design cannot be observational and prospective Mahajan
(\citeproc{ref-Mahajan2015}{2015}), meaning it cannot give insights over
a long time and perceive the dynamic change. The causal conclusion could
not be demonstrated through it. Second, the study sample was not big,
and it was also a convenient sample. This means the sample cannot
accurately represent a broader population, and the selection bias also
existed. It can make findings lack generalizability. Third, there is a
limitation about the scales. The job satisfaction scale used was too
old, which may not be accurate enough for modern people and suitable for
the current work situation, which may not be comprehensive. To handle
these problems, Spector (\citeproc{ref-Spector2019}{2019}) suggested
that a cross-sectional design could be utilized well if we can establish
four elements: establishing covariation, temporal precedence, ruling out
alternatives, and explanatory mechanism. Moreover, maybe adding some
time elements to the scale may help reduce bias from occasion factors.
Additionally, future research should avoid using convenience samples and
expand the sample size. Last but not least, future researchers could
find a better job satisfaction scale to help evaluate employee job
satisfaction.

\section*{References}\label{references}
\addcontentsline{toc}{section}{References}

\phantomsection\label{refs}
\begin{CSLReferences}{1}{0}
\bibitem[\citeproctext]{ref-Aslaniyan2013}
Aslaniyan, M., \& Moghaddam, M. S. (2013). \emph{A review and modeling
on job satisfaction in {Zahedan} municipality district {No}.}
\emph{8}(4).

\bibitem[\citeproctext]{ref-Atoum2018}
Atoum, A. Y., \& Al-Shoboul, R. A. (2018). Emotional support and its
relationship to {Emotional} intelligence. \emph{Advances in Social
Sciences Research Journal}, \emph{5}(1).
\url{https://doi.org/10.14738/assrj.51.4095}

\bibitem[\citeproctext]{ref-AugustoLanda2008}
Augusto Landa, J. M., López-Zafra, E., Berrios Martos, M. P., \&
Aguilar-Luzón, M. D. C. (2008). The relationship between emotional
intelligence, occupational stress and health in nurses: {A}
questionnaire survey. \emph{International Journal of Nursing Studies},
\emph{45}(6), 888--901.
\url{https://doi.org/10.1016/j.ijnurstu.2007.03.005}

\bibitem[\citeproctext]{ref-Aziri2011}
Aziri, B. (2011). \emph{{JOB SATISFACTION}: {A LITERATURE REVIEW}}.
\emph{3}(4).

\bibitem[\citeproctext]{ref-BambooHR2016}
BambooHR. (2016, April 20). \emph{Job satisfaction of {U}.{S}. Employees
by recognition program 2016}. Statista.
\url{https://www.statista.com/statistics/745431/job-satisfaction-of-us-employees-by-recognition-program/}

\bibitem[\citeproctext]{ref-Brayfield1951}
Brayfield, A. H., \& Rothe, H. F. (1951). An index of job satisfaction.
\emph{Journal of Applied Psychology}, \emph{35}(5), 307--311.
\url{https://doi.org/10.1037/h0055617}

\bibitem[\citeproctext]{ref-Capgemini2019}
Capgemini. (2019, November 13). \emph{Organizations assessing and hiring
based on emotional intelligence skills 2019}. Statista.
\url{https://www.statista.com/statistics/1074180/share-organizations-assessing-hiring-emotional-intelligence-worldwide/}

\bibitem[\citeproctext]{ref-Chan2002}
Chan, S. (2002). Factors influencing nursing leadership effectiveness in
{Hong Kong}. \emph{Journal of Advanced Nursing}, \emph{38}(6), 615--623.
\url{https://doi.org/10.1046/j.1365-2648.2002.02229.x}

\bibitem[\citeproctext]{ref-ChiokFoongLoke2001}
Chiok Foong Loke, J. (2001). Leadership behaviours: Effects on job
satisfaction, productivity and organizational commitment. \emph{Journal
of Nursing Management}, \emph{9}(4), 191--204.
\url{https://doi.org/10.1046/j.1365-2834.2001.00231.x}

\bibitem[\citeproctext]{ref-Connelly2016}
Connelly, L. M. (Sep/Oct 2016). Cross-{Sectional Survey Research}.
\emph{Medsurg Nursing}, \emph{25}(5), 369--370.
\url{https://www.proquest.com/docview/1827241811/citation/FB72CCC6A5414408PQ/1}

\bibitem[\citeproctext]{ref-Cortina1993}
Cortina, J. M. (1993). What is coefficient alpha? {An} examination of
theory and applications. \emph{Journal of Applied Psychology},
\emph{78}(1), 98--104. \url{https://doi.org/10.1037/0021-9010.78.1.98}

\bibitem[\citeproctext]{ref-Cumbey1998}
Cumbey, D. A., \& Alexander, J. W. (1998). The {Relationship} of {Job
Satisfaction} with {Organizational Variables} in {Public Health
Nursing}: \emph{The Journal of Nursing Administration}, \emph{28}(5),
39--46. \url{https://doi.org/10.1097/00005110-199805000-00007}

\bibitem[\citeproctext]{ref-Danish2010}
Danish, R. Q., \& Usman, A. (2010). Impact of {Reward} and {Recognition}
on {Job Satisfaction} and {Motivation}: {An Empirical} study from
{Pakistan}. \emph{International Journal of Business and Management},
\emph{5}(2), p159. \url{https://doi.org/10.5539/ijbm.v5n2p159}

\bibitem[\citeproctext]{ref-Darvishmotevali2018}
Darvishmotevali, M., Altinay, L., \& De Vita, G. (2018). Emotional
intelligence and creative performance: {Looking} through the lens of
environmental uncertainty and cultural intelligence. \emph{International
Journal of Hospitality Management}, \emph{73}, 44--54.
\url{https://doi.org/10.1016/j.ijhm.2018.01.014}

\bibitem[\citeproctext]{ref-Davis-Kean2005}
Davis-Kean, P. E. (2005). The {Influence} of {Parent Education} and
{Family Income} on {Child Achievement}: {The Indirect Role} of {Parental
Expectations} and the {Home Environment}. \emph{Journal of Family
Psychology}, \emph{19}(2), 294--304.
\url{https://doi.org/10.1037/0893-3200.19.2.294}

\bibitem[\citeproctext]{ref-Edelman2018}
Edelman, P., \& van Knippenberg, D. (2018). Emotional intelligence,
management of subordinate's emotions, and leadership effectiveness.
\emph{Leadership \& Organization Development Journal}, \emph{39}(5),
592--607. \url{https://doi.org/10.1108/LODJ-04-2018-0154}

\bibitem[\citeproctext]{ref-Emmanuel2021}
Emmanuel, T., Maupong, T., Mpoeleng, D., Semong, T., Mphago, B., \&
Tabona, O. (2021). A survey on missing data in machine learning.
\emph{Journal of Big Data}, \emph{8}(1), 140.
\url{https://doi.org/10.1186/s40537-021-00516-9}

\bibitem[\citeproctext]{ref-Faragher2005}
Faragher, E., Cass, M., \& Cooper, C. (2005). The relationship between
job satisfaction and health: A meta-analysis. \emph{Occupational and
Environmental Medicine}, \emph{62}(2), 105--112.
\url{https://doi.org/10.1136/oem.2002.006734}

\bibitem[\citeproctext]{ref-Hartman1999}
Hartman, L. (1999). A psychological analysis of leadership
effectiveness. \emph{Strategy \& Leadership}, \emph{27}(6), 30--32.
\url{https://doi.org/10.1108/eb054651}

\bibitem[\citeproctext]{ref-Hassan2018}
Hassan, A., Gallear, D., \& Sivarajah, U. (2018). Critical factors
affecting leadership: A higher education context. \emph{Transforming
Government: People, Process and Policy}, \emph{12}(1), 110--130.
\url{https://doi.org/10.1108/TG-12-2017-0075}

\bibitem[\citeproctext]{ref-Helm2013}
Helm, S. (2013). A {Matter} of {Reputation} and {Pride}: {Associations}
between {Perceived External Reputation}, {Pride} in {Membership}, {Job
Satisfaction} and {Turnover Intentions}. \emph{British Journal of
Management}, \emph{24}(4), 542--556.
\url{https://doi.org/10.1111/j.1467-8551.2012.00827.x}

\bibitem[\citeproctext]{ref-Hu2023}
Hu, X., Li, R. Y. M., Kumari, K., Ben Belgacem, S., Fu, Q., Khan, M. A.,
\& Alkhuraydili, A. A. (2023). Relationship between {Green Leaders}'
{Emotional Intelligence} and {Employees}' {Green Behavior}: {A PLS-SEM
Approach}. \emph{Behavioral Sciences}, \emph{13}(1, 1), 25.
\url{https://doi.org/10.3390/bs13010025}

\bibitem[\citeproctext]{ref-Inuwa2016}
Inuwa, M. (2016). Job {Satisfaction} and {Employee Performance}: {An
Empirical Approach}. \emph{The Millennium University Journal},
\emph{1}(1), 90--103. \url{https://doi.org/10.58908/tmuj.v1i1.10}

\bibitem[\citeproctext]{ref-Irabor2019}
Irabor, I. E., \& Okolie, U. C. (2019). A {Review} of {Employees}' {Job
Satisfaction} and its {Affect} on their {Retention}. \emph{Annals of
Spiru Haret University. Economic Series}, \emph{19}(2), 93--114.
\url{https://doi.org/10.26458/1924}

\bibitem[\citeproctext]{ref-Jacobsen2022}
Jacobsen, C. B., Andersen, L. B., Bøllingtoft, A., \& Eriksen, T. L. M.
(2022). Can {Leadership Training Improve Organizational Effectiveness}?
{Evidence} from a {Randomized Field Experiment} on {Transformational}
and {Transactional Leadership}. \emph{Public Administration Review},
\emph{82}(1), 117--131. \url{https://doi.org/10.1111/puar.13356}

\bibitem[\citeproctext]{ref-Jahro2022}
Jahro, L., Harapan, E., \& Tahrun, T. (2022). The {Effect} of
{Leadership Effectiveness} and {Interpersonal Communication} on {Teacher
Job Satisfaction}. \emph{Journal of Social Work and Science Education},
\emph{2}(3), 219--226. \url{https://doi.org/10.52690/jswse.v2i3.252}

\bibitem[\citeproctext]{ref-Judge2004}
Judge, T. A., Colbert, A. E., \& Ilies, R. (2004). Intelligence and
{Leadership}: {A Quantitative Review} and {Test} of {Theoretical
Propositions}. \emph{Journal of Applied Psychology}, \emph{89}(3),
542--552. \url{https://doi.org/10.1037/0021-9010.89.3.542}

\bibitem[\citeproctext]{ref-Kardam2012}
Kardam, B. L., \& Rangnekar, S. (2012). Job {Satisfaction}:
{Investigating} the {Role} of {Experience} \& {Education}.
\emph{Researchers World}, \emph{3}(4), 16--22.
\url{https://www.proquest.com/docview/1285124675/abstract/5421AC1301E24574PQ/1}

\bibitem[\citeproctext]{ref-Katyal2017}
Katyal, S., \& Awasthi, E. (2017). Gender {Differences} in {Emotional
Intelligence Among Adolescents} of {Chandigarh}. \emph{Journal of Human
Ecology}, \emph{17}(2), 153--155.
\url{https://doi.org/10.1080/09709274.2005.11905771}

\bibitem[\citeproctext]{ref-Kerr2006}
Kerr, R., Garvin, J., Heaton, N., \& Boyle, E. (2006). Emotional
intelligence and leadership effectiveness. \emph{Leadership \&
Organization Development Journal}, \emph{27}(4), 265--279.
\url{https://doi.org/10.1108/01437730610666028}

\bibitem[\citeproctext]{ref-Khan2012}
Khan, A. H., Nawaz, M. M., Aleem, M., \& Hamed, W. (2012). Impact of job
satisfaction on employee performance: {An} empirical study of autonomous
{Medical Institutions} of {Pakistan}. \emph{African Journal of Business
Management}, \emph{6}(7), pp. 2697--2705.
\url{https://doi.org/10.5897/AJBM11.2222}

\bibitem[\citeproctext]{ref-Levine2006}
Levine, J. M., \& Moreland, R. L. (Eds.). (2006). \emph{Small groups:
Key readings}. Psychology Press.

\bibitem[\citeproctext]{ref-Liu2018}
Liu, Y., Chan, C., Zhao, C., \& Liu, C. (2018). Unpacking knowledge
management practices in {China}: Do institution, national and
organizational culture matter? \emph{Journal of Knowledge Management},
\emph{23}(4), 619--643. \url{https://doi.org/10.1108/JKM-07-2017-0260}

\bibitem[\citeproctext]{ref-Locke1969}
Locke, E. A. (1969). What is job satisfaction? \emph{Organizational
Behavior and Human Performance}, \emph{4}(4), 309--336.
\url{https://doi.org/10.1016/0030-5073(69)90013-0}

\bibitem[\citeproctext]{ref-Lone2018}
Lone, M. A., \& Lone, A. H. (2018). Does {Emotional Intelligence Predict
Leadership Effectiveness}? {An Exploration} in {Non-Western Context}.
\emph{South Asian Journal of Human Resources Management}, \emph{5}(1),
28--39. \url{https://doi.org/10.1177/2322093718766806}

\bibitem[\citeproctext]{ref-Lowenhaupt2014}
Lowenhaupt, R. J. (2014). The language of leadership: Principal rhetoric
in everyday practice. \emph{Journal of Educational Administration},
\emph{52}(4), 446--468. \url{https://doi.org/10.1108/JEA-11-2012-0118}

\bibitem[\citeproctext]{ref-Lu2011}
Lu, L., Kao, S.-F., Siu, O.-L., \& Lu, C.-Q. (2011). Work {Stress},
{Chinese Work Values}, and {Work Well-Being} in the {Greater China}.
\emph{The Journal of Social Psychology}, \emph{151}(6), 767--783.
\url{https://doi.org/10.1080/00224545.2010.538760}

\bibitem[\citeproctext]{ref-Lund2003}
Lund, D. B. (2003). Organizational culture and job satisfaction.
\emph{Journal of Business \& Industrial Marketing}, \emph{18}(3),
219--236. \url{https://doi.org/10.1108/0885862031047313}

\bibitem[\citeproctext]{ref-Luu2010}
Luu, L., \& Hattrup, K. (2010). An {Investigation} of {Country
Differences} in the {Relationship Between Job Satisfaction} and
{Turnover Intentions}. \emph{Applied H.R.M. Research}, \emph{12}(1),
17--39.
\url{https://www.proquest.com/docview/864542817/citation/6A4E5769ECBC4F0BPQ/1}

\bibitem[\citeproctext]{ref-Mahajan2015}
Mahajan, A. (2015). Limitations of cross-sectional studies.
\emph{Neurology India}, \emph{63}(6), 1006--1007.
\url{https://doi.org/10.4103/0028-3886.170110}

\bibitem[\citeproctext]{ref-Marshall2010}
Marshall, G., \& Jonker, L. (2010). An introduction to descriptive
statistics: {A} review and practical guide. \emph{Radiography},
\emph{16}(4), e1--e7. \url{https://doi.org/10.1016/j.radi.2010.01.001}

\bibitem[\citeproctext]{ref-Martinez1997}
Martinez, M. N. (1997). The smarts that count. \emph{HR Magazine},
\emph{42}(11), 72.
\url{https://kean.idm.oclc.org/login?url=https://search.ebscohost.com/login.aspx?direct=true&AuthType=cookie,ip,url,cpid&custid=keaninf&db=b9h&AN=9712072813&site=ehost-live&scope=site}

\bibitem[\citeproctext]{ref-Mayer1990}
Mayer, J. D., DiPaolo, M., \& Salovey, P. (1990). Perceiving {Affective
Content} in {Ambiguous Visual Stimuli}: {A Component} of {Emotional
Intelligence}. \emph{Journal of Personality Assessment},
\emph{54}(3--4), 772--781.
\url{https://doi.org/10.1080/00223891.1990.9674037}

\bibitem[\citeproctext]{ref-Miao2016}
Miao, C., Humphrey, R. H., \& Qian, S. (2016). Leader emotional
intelligence and subordinate job satisfaction: {A} meta-analysis of
main, mediator, and moderator effects. \emph{Personality and Individual
Differences}, \emph{102}, 13--24.
\url{https://doi.org/10.1016/j.paid.2016.06.056}

\bibitem[\citeproctext]{ref-Nguyen2022}
Nguyen, T. L., Nguyen, H. A. M., Luu, P. T. N., Le, M. A., Nguyen, T. A.
T., \& Nguyen, N. T. (2022). Leadership and {Communication Skills
Towards Emotional Intelligence}: {A Case Study} of {FPT University} in
{Vietnam}. \emph{The Journal of Asian Finance, Economics and Business},
\emph{9}(5), 53--61.
\url{https://doi.org/10.13106/jafeb.2022.vol9.no5.0053}

\bibitem[\citeproctext]{ref-Pau2004}
Pau, A. K. H., Croucher, R., Sohanpal, R., Muirhead, V., \& Seymour, K.
(2004). Emotional intelligence and stress coping in dental
undergraduates --- a qualitative study. \emph{British Dental Journal},
\emph{197}(4), 205--209. \url{https://doi.org/10.1038/sj.bdj.4811573}

\bibitem[\citeproctext]{ref-Poturak2020}
Poturak, M., Mekić, E., Hadžiahmetović, N., \& Budur, T. (2020).
Effectiveness of {Transformational Leadership} among {Different
Cultures}. \emph{International Journal of Social Sciences and
Educational Studies}, \emph{7}(3).
\url{https://doi.org/10.23918/ijsses.v7i3p119}

\bibitem[\citeproctext]{ref-Ramesar2009}
Ramesar, S., Koortzen, P., \& Oosthuizen, R. M. (2009). The relationship
between emotional intelligence and stress management. \emph{SA Journal
of Industrial Psychology}, \emph{35}(1), 10 pages.
\url{https://doi.org/10.4102/sajip.v35i1.443}

\bibitem[\citeproctext]{ref-Rauf2013}
Rauf, F. H. A., Tarmidi, M., Omar, M., \& Yaaziz, N. N. R. (2013).
Personal, {Family} and {Academic Factors} towards {Emotional
Intelligence}: {A Case Study}. \emph{Emotional Intelligence}.

\bibitem[\citeproctext]{ref-Rawat2023}
Rawat, S., Deshpande, A. P., Boe, O., \& Piotrowski, A. (2023).
Understanding {Leadership Effectiveness} in the wake of challenges: A
leadership competency model. \emph{HUMAN REVIEW. International
Humanities Review / Revista Internacional de Humanidades}, \emph{12}(1).
\url{https://doi.org/10.37467/revhuman.v12.3498}

\bibitem[\citeproctext]{ref-Rosete2005}
Rosete, D., \& Ciarrochi, J. (2005-07-01Z). Emotional intelligence and
its relationship to workplace performance outcomes of leadership
effectiveness. \emph{Leadership \&Amp; Organization Development
Journal}, \emph{26}(5), 388--399.
\url{https://doi.org/10.1108/01437730510607871}

\bibitem[\citeproctext]{ref-Sadeghi2012}
Sadeghi, A., Akmaliah, Z., \& lope pihie, Z. (2012). Transformational
{Leadership} and {Its Predictive Effects} on {Leadership Effectiveness}.
\emph{International Journal of Business and Social Science}, \emph{3}.

\bibitem[\citeproctext]{ref-SanLam2012}
San Lam, C., \& O'Higgins, E. R. E. (2012). Enhancing employee outcomes:
{The} interrelated influences of managers' emotional intelligence and
leadership style. \emph{Leadership \& Organization Development Journal},
\emph{33}(2), 149--174. \url{https://doi.org/10.1108/01437731211203465}

\bibitem[\citeproctext]{ref-Schuetz2017}
Schuetz, A. (2017). Effective {Leadership} and its {Impact} on an
{Organisation}'s {Success}. \emph{Journal of Corporate Responsibility
and Leadership}, \emph{3}(3), 73.
\url{https://doi.org/10.12775/JCRL.2016.017}

\bibitem[\citeproctext]{ref-Spector2019}
Spector, P. E. (2019). Do {Not Cross Me}: {Optimizing} the {Use} of
{Cross-Sectional Designs}. \emph{Journal of Business and Psychology},
\emph{34}(2), 125--137. \url{https://doi.org/10.1007/s10869-018-09613-8}

\bibitem[\citeproctext]{ref-Stratton2021}
Stratton, S. J. (2021). Population {Research}: {Convenience Sampling
Strategies}. \emph{Prehospital and Disaster Medicine}, \emph{36}(4),
373--374. \url{https://doi.org/10.1017/S1049023X21000649}

\bibitem[\citeproctext]{ref-Sy2006}
Sy, T., Tram, S., \& O'Hara, L. A. (2006). Relation of employee and
manager emotional intelligence to job satisfaction and performance.
\emph{Journal of Vocational Behavior}, \emph{68}(3), 461--473.
\url{https://doi.org/10.1016/j.jvb.2005.10.003}

\bibitem[\citeproctext]{ref-Tanveer2019}
Tanveer, Y., Tariq, A., Akram, U., \& Bilal, M. (2019). Tactics of
handling interpersonal conflict through emotional intelligence.
\emph{Int. J. Information Systems and Change Management},
\emph{11}(3/4).

\bibitem[\citeproctext]{ref-Thekedam2010}
Thekedam, J. S. (2010). A {Study} of {Job Satisfaction} and {Factors}
that {Influence} it. \emph{Management and Labour Studies}, \emph{35}(4),
407--417. \url{https://doi.org/10.1177/0258042X1003500401}

\bibitem[\citeproctext]{ref-Ursachi2015}
Ursachi, G., Horodnic, I. A., \& Zait, A. (2015). How {Reliable} are
{Measurement Scales}? {External Factors} with {Indirect Influence} on
{Reliability Estimators}. \emph{Procedia Economics and Finance},
\emph{20}, 679--686. \url{https://doi.org/10.1016/S2212-5671(15)00123-9}

\bibitem[\citeproctext]{ref-VanRooy2005}
Van Rooy, D. L., Alonso, A., \& Viswesvaran, C. (2005). Group
differences in emotional intelligence scores: Theoretical and practical
implications. \emph{Personality and Individual Differences},
\emph{38}(3), 689--700. \url{https://doi.org/10.1016/j.paid.2004.05.023}

\bibitem[\citeproctext]{ref-Vilkinas2009}
Vilkinas, T., Shen, J., \& Cartan, G. (2009). Predictors of leadership
effectiveness for {Chinese} managers. \emph{Leadership \& Organization
Development Journal}, \emph{30}(6), 577--590.
\url{https://doi.org/10.1108/01437730910981944}

\bibitem[\citeproctext]{ref-Vlachos2013}
Vlachos, P. A., Panagopoulos, N. G., \& Rapp, A. A. (2013). Feeling
{Good} by {Doing Good}: {Employee CSR-Induced Attributions}, {Job
Satisfaction}, and the {Role} of {Charismatic Leadership}. \emph{Journal
of Business Ethics}, \emph{118}(3), 577--588.
\url{https://doi.org/10.1007/s10551-012-1590-1}

\bibitem[\citeproctext]{ref-Wang2016}
Wang, Y., Xie, G., \& Cui, X. (2016). Effects of {Emotional
Intelligence} and {Selfleadership} on {Students}' {Coping} with
{Stress}. \emph{Social Behavior and Personality: An International
Journal}, \emph{44}(5), 853--864.
\url{https://doi.org/10.2224/sbp.2016.44.5.853}

\bibitem[\citeproctext]{ref-Waqas2014}
Waqas, A., Bashir, U., Sattar, M. F., Abdullah, H. M., Hussain, I.,
Anjum, W., Aftab Ali, M., \& Arshad, R. (2014). Factors {Influencing Job
Satisfaction} and {Its Impact} on {Job Loyalty}. \emph{International
Journal of Learning and Development}, \emph{4}(2).
\url{https://doi.org/10.5296/ijld.v4i2.6095}

\bibitem[\citeproctext]{ref-Wong2001}
Wong, C.-S., Wong, Y., Hui, C., \& Law, K. S. (2001). The significant
role of {Chinese} employees' organizational commitment: Implications for
managing employees in {Chinese} societies. \emph{Journal of World
Business}, \emph{36}(3), 326--340.
\url{https://doi.org/10.1016/S1090-9516(01)00058-X}

\bibitem[\citeproctext]{ref-Yadav2019}
Yadav, R., \& Lata, P. (2019). \emph{Role of {Emotional Intelligence} in
{Effective Leadership}}. \emph{7}(2).

\bibitem[\citeproctext]{ref-Yahaya2016}
Yahaya, R., \& Ebrahim, F. (2016). Leadership styles and organizational
commitment: Literature review. \emph{Journal of Management Development},
\emph{35}(2), 190--216. \url{https://doi.org/10.1108/JMD-01-2015-0004}

\bibitem[\citeproctext]{ref-Yukl2019}
Yukl, G., Mahsud, R., Prussia, G., \& Hassan, S. (2019). Effectiveness
of broad and specific leadership behaviors. \emph{Personnel Review},
\emph{48}. \url{https://doi.org/10.1108/PR-03-2018-0100}

\bibitem[\citeproctext]{ref-Zeidner2004}
Zeidner, M., Matthews, G., \& Roberts, R. D. (2004). Emotional
{Intelligence} in the {Workplace}: {A Critical Review}. \emph{Applied
Psychology}, \emph{53}(3), 371--399.
\url{https://doi.org/10.1111/j.1464-0597.2004.00176.x}

\bibitem[\citeproctext]{ref-Zhang2019}
Zhang, X., Kaiser, M., Nie, P., \& Sousa-Poza, A. (2019年9月26日). Why
are {Chinese} workers so unhappy? {A} comparative cross-national
analysis of job satisfaction, job expectations, and job attributes.
\emph{PLOS ONE}, \emph{14}(9), e0222715.
\url{https://doi.org/10.1371/journal.pone.0222715}

\end{CSLReferences}

\appendix

\subsection{Consent to Participate in
Research}\label{consent-to-participate-in-research}

Consent to Participate in Research

Research Title: Leader's Emotional Intelligence and Job Satisfaction:
The Mediating Role of Leadership Effectiveness

Research Assistant: Fu, Yangyang

Departments: Department of Psychology, College of Liberal Arts
Institution: Wenzhou-Kean University

Contact: +86 18072333510; Email: fuya@kean.edu

Research risk:

Participation in this study is purely voluntary. Participants can choose
to stop at any time without negative consequences. Their participation
will not involve risks or discomforts more than are common in everyday
life. The questions are designed to be non-invasive and respect
participants' privacy.

Confidentiality and Records:

Any information collected for this study will be kept strictly
confidential and used for research purposes only. All data, including
questionnaires and tests, will be identified by the code assigned to
each participant by the online portal, not by the participant's name.
The file containing this assignment will be stored in encrypted form and
only researchers associated with this study will have access to this
file and the raw data. Study data will not be disclosed to any third
party unless we have the express approval of the participant.
Participants have access to raw data from this process.

Before starting to participate in this study, you agree to:

• You are over 18 years old;

• You are an employee or intern;

• Your participation in this study is entirely voluntary.

• You can leave your studies at any time; if you decide to stop
participating in the study, you will not be penalized and you will not
lose any benefits to which you are otherwise entitled.

Informed Consent:

I have read and understood the information provided and have had the
opportunity to ask questions. My participation in the survey is
voluntary and I may withdraw at any time without reason. If you agree to
the above content, please click the ``Agree'' button below to start
answering the questions. If you do not agree with the above content,
please click the ``Disagree'' button below to terminate this survey
project.

□ Agree □ Disagree

\subsection{Questionnaire}\label{questionnaire}

\subsubsection{Leaders' Emotional
Intelligence}\label{leaders-emotional-intelligence-1}

(1=Strongly Disagree 2=Disagree 3=Netural 4=Agree 5=Strongly Agree,
higher the score means higher Leaders' Emotional Intelligence)

LEQ1.My leader has a good sense of why he has certain feelings most of
the time.

LEQ2.My leader has a good understanding of his own emotions.

LEQ3.My leader really understands what he feels.

LEQ4.My leader always knows whether or not he is happy.

LEQ5.My leader always knows his friends' emotions from their behavior.

LEQ6.My leader is a good observer of others' emotions.

LEQ7.My leader is sensitive to the feelings and emotions of others.

LEQ8.My leader is good understanding of the emotions of people around
him.

LEQ9.My leader always set goals for himself and then try his best to
achieve them.

LEQ10.My leader always tell himself that he is a competent person.

LEQ11.My leader is a self-motivated person.

LEQ12.My leader always encourages himself to try his best.

LEQ13.My leader is able to control his temper and handle difficulties
rationally.

LEQ14.My leader is quite capable of controlling his own emotions.

LEQ15.My leader always calms down quickly when he is very angry.

LEQ16.My leader has good control of his own emotions.

\subsubsection{Leadership effectiveness
scale}\label{leadership-effectiveness-scale-1}

(1=Strongly Disagree 2=Disagree 3=Netural 4=Agree 5=Strongly Agree,
higher the score means higher Leaders' Effectiveness)

LE1.I feel at ease with my supervisor when asking questions.

LE2.My supervisor asks me how I am doing on a regular basis.

LE3.I feel I am treated in a fair manner.

LE4.My supervisor supports me when I need help.

LE5.Keeping my supervisor informed, I can take initiatives.

LE6.We are involved as a team in solving problems related to our work.

LE7.We are involved as a team in decisions made that affect our work.

LE8.I am involved as an individual in solving problems related to our
work.

LE9.I am involved as an individual in decisions made that affect my work

\subsubsection{Job Satisfaction Scale}\label{job-satisfaction-scale-1}

(1=Strongly Disagree 2=Disagree 3=Netural 4=Agree 5=Strongly Agree,
higher the score means higher Job Satisfaction)

JS1.My job is like a hobby to me.

JS2.My job is usually interesting enough to keep me from getting bored.

JS3.It seems that my friends are more interested in their jobs.

JS4.I consider my job rather unpleasant.

JS5.I enjoy my work more than my leisure time.

JS6.I am often bored with my job.

JS7.I feel fairly well satisfied with my present job.

JS8.Most of the time I have to force myself to go to work

JS9.I am satisfied with my job for the time being.

JS10.I feel that my job is no more interesting than others I could get.

JS11.I definitely dislike my work.

JS12.I feel that I am happier in my work than most other people.

JS13.Most days I am enthusiastic about my work.

JS14.Each day of work seems like it will never end.

JS15.I like my job better than the average worker does.

JS16.My job is pretty uninteresting.

JS17.I find real enjoyment in my work.

JS18.I am disappointed that I ever took this job.

\subsubsection{Demographic Questions}\label{demographic-questions}

1.Your Gender:

○Male

○Female

○Prefer not to tell

2.Designation:

○Intern

○Regular Employee

○Coordinator

○Supervisor

○Senior Manager

○Managing Director

○Others

3.Your current industry:

○Education

○Manufacturing Industry

○Construction Industry

○Internet

○Electric gas

○others

\section{Title for Appendix}\label{title-for-appendix}






\end{document}
